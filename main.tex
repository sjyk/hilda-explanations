\documentclass[sigconf]{acmart}

\usepackage{enumitem}
\usepackage{framed}
\usepackage[11pt]{moresize}
\usepackage{cprotect}
\usepackage{enumitem}
\usepackage{listings}
\usepackage{amstext}
\usepackage{amstext}
\usepackage{pdfpages}
\usepackage{alltt}
\usepackage{epstopdf}
\usepackage{xspace,colortbl}
\usepackage[USenglish]{babel}
\usepackage{multirow}
\usepackage{url}
\usepackage{graphicx}%%
\usepackage{amssymb}
\usepackage{fmtcount}
\usepackage{amsfonts}
\usepackage{xspace}
\usepackage{amsmath}
\usepackage{multirow}
\usepackage[mathscr]{eucal}
%\usepackage{psfrag}
\usepackage{colortbl}
\usepackage{times}


\usepackage{bm}
%\usepackage[nospace]{cite}
\usepackage{csquotes}
\usepackage{enumitem}

\lstset{basicstyle=\small,breaklines=true,language=Python,belowcaptionskip=.1\baselineskip}

\usepackage{balance}

%\linespread{0.99}

\copyrightyear{2017} 
\acmYear{2017} 
\setcopyright{acmcopyright}
\acmConference{HILDA'17}{May 14, 2017}{Chicago, IL, USA}\acmPrice{15.00}\acmDOI{http://dx.doi.org/10.1145/3077257.3077271}
\acmISBN{978-1-4503-5029-7/17/05}

\begin{document}

\setlength{\belowdisplayskip}{3pt} \setlength{\belowdisplayshortskip}{3pt}
\setlength{\abovedisplayskip}{3pt} \setlength{\abovedisplayshortskip}{3pt}
\setlength{\belowcaptionskip}{-10pt}
\selectfont

\newcommand{\cond}{\textrm{pred}\xspace}
\newcommand{\dataset}{data set\xspace}
\newcommand{\datasets}{data sets\xspace}
\newcommand{\spview}{\textsf{SPView}\xspace}
\newcommand{\fjview}{\textsf{FJView}\xspace}
\newcommand{\aggview}{\textsf{AggView}\xspace}
\newcommand{\hashfunc}[1]{\textsf{hash}(#1)\xspace}
\newcommand{\hashop}{\textsf{hash}\xspace}
\newcommand{\nsc}{\textsf{NormalizedSC}\xspace}
\newcommand{\rsc}{\textsf{RawSC}\xspace}

\newcommand{\avgfunc}{\ensuremath{\texttt{avg} }\xspace}
\newcommand{\maxfunc}{\ensuremath{\texttt{max} }\xspace}
\newcommand{\minfunc}{\ensuremath{\texttt{min} }\xspace}
\newcommand{\histfunc}{\ensuremath{\texttt{histogram\_numeric} }\xspace}
\newcommand{\countfunc}{\ensuremath{\texttt{count}}\xspace}
\newcommand{\sumfunc}{\ensuremath{\texttt{sum} }\xspace}
\newcommand{\varfunc}{\ensuremath{\texttt{var} }\xspace}
\newcommand{\stdfunc}{\ensuremath{\texttt{std} }\xspace}
\newcommand{\covfunc}{\ensuremath{\texttt{cov} }\xspace}
\newcommand{\corrfunc}{\ensuremath{\texttt{corr} }\xspace}
\newcommand{\medfunc}{\ensuremath{\texttt{median} }\xspace}
\newcommand{\percfunc}{\ensuremath{\texttt{percentile} }\xspace}
\newcommand{\havingfunc}{\ensuremath{\texttt{HAVING} }\xspace}
\newcommand{\selectfunc}{\ensuremath{\texttt{select} }\xspace}
\newcommand{\ratio}{\ensuremath{\rho }\xspace}


\newcommand{\insertion}{\ensuremath{\texttt{INSERT} }\xspace}
\newcommand{\update}{\ensuremath{\texttt{UPDATE} }\xspace}
\newcommand{\delete}{\ensuremath{\texttt{DELETE} }\xspace}

\newcommand{\sysfull}{Partition Aware Local Model\xspace}
\newcommand{\sys}{PALM\xspace}
\newcommand{\sysnospace}{PALM}

\newcommand{\tbl}[1]{\textsf{#1}\xspace}
\newcommand{\field}[1]{\textsf{#1}\xspace}
\newcommand{\cost}{\textrm{cost}\xspace}
\newcommand{\ans}{\textsf{ans}\xspace}
\newcommand{\dans}{\Delta\textsf{ans}\xspace}
\newcommand{\cqp}{correction query processing\xspace}
\newcommand{\Cqp}{Correction query processing\xspace}

\newcommand{\gray}[1]{{{\textcolor{lightgray}{\{#1\}}}\xspace}}
\newcommand{\ewu}[1]{{{\textcolor{red}{\{ewu: #1\}}}\xspace}}
\newcommand{\reminder}[1]{{{\textcolor{magenta}{\{\{\bf #1\}\}}}\xspace}}
\newcommand{\specialcell}[2][c]{%
  \begin{tabular}[#1]{@{}c@{}}#2\end{tabular}}

\def\ojoin{\setbox0=\hbox{$\bowtie$}%
  \rule[-.02ex]{.25em}{.4pt}\llap{\rule[\ht0]{.25em}{.4pt}}}
\def\leftouterjoin{\mathbin{\ojoin\mkern-5.8mu\bowtie}}
\def\rightouterjoin{\mathbin{\bowtie\mkern-5.8mu\ojoin}}
\def\fullouterjoin{\mathbin{\ojoin\mkern-5.8mu\bowtie\mkern-5.8mu\ojoin}}


\newcommand{\stitle}[1]{\vspace{0.5em}\noindent\textbf{#1}}


%\setlength{\belowcaptionskip}{-10pt}

%\newcommand{\reminder}[1] {}

%\input{coverletter.tex}

%\title{ActiveClean: Progressive Data Cleaning For Convex Data Analytics}
%\title{Towards Machine Learning Model Explanations That Are Useful For Debugging}
\title{\sys: Machine Learning Explanations For Iterative Debugging}

%\numberofauthors{2}


\author{Sanjay Krishnan}
\affiliation{\institution{UC Berkeley}}
\email{sanjaykrishnan@berkeley.edu}

\author{Eugene Wu}
\affiliation{\institution{Columbia University}}
\email{ewu@cs.columbia.edu}


%\fontsize{9pt}{11pt}
%\selectfont




\begin{abstract}
Emerging applications for machine learning in computer vision and natural language processing are using ever more complex and high-dimensional models.
To effectively debug such applications,  we argue that developers need a form of data provenance, namely, where they can isolate a small set of training examples that explain why a new example was predicted in a certain way.
Linking a prediction to responsible training data allows a data scientist to look for inconsistencies between training and test data, and understand the structure of a complex model.
The challenge is that in most models are coupled in nature where every training example contributes to a prediction in some way.
We present \sysfull (\sys), which approximates a complex model (e.g., a deep neural network) using a two-part surrogate model: a meta-model that partitions the training data, and a set of sub-models that approximate the patterns within each partition.
The user can use \sys to guide data exploration to identify problematic examples in the training dataset.
Queries to \sys are nearly 30x faster than nearest neighbor queries for identifying relevant data, which is a key property for interactive applications.
\end{abstract}

\maketitle

\pagenumbering{gobble}


\section{Introduction}\label{intro}
The availability of data and vast cloud-based computational resources is allowing both industry and academia to apply Machine Learning in many new domains.
Learning-based components are increasingly in the critical path of important software systems such as in fraud detection, product recommendation, robotics and control, and machine vision.
Several breakthroughs in recent years have resulted in improved computational efficiency~\cite{feng2012towards, tensor, recht2011hogwild, crotty2014tupleware} as well as ease of development for learning applications~\cite{hellerstein2012madlib, keystone, kraska2013mlbase}.
This research has culminated in the development of several open-source toolkits with highly efficient distributed optimization libraries, e.g., TensorFlow, MLlib, and CAFFE, or in other words--training a statistical model has never been easier.

However, surveys of data scientists, the ones who have ostensibly benefited the most from these breakthroughs, paint a very different picture~\cite{kandel2012, krishnan2016hilda}.
Building and debugging data-driven applications remains to be time-consuming and effort-intensive~\cite{sculley2014machine}.
When a prediction is unexpected, it is unclear if this is due to a data error (i.e., an inconsistent value representation), a model error (i.e., the model class cannot represent such a relationship), or an approximation error (i.e., such examples were not encountered in sufficient number during training).
When models are opaque, diagnosing such problems is difficult, so \emph{interpretability} is often touted as a solution \cite{?}.

But, interpretability means different things to different people. To an end-user, this means explain why I received a given prediction in terms of understandable features. However, an ML developer cares less about understandable features than an explanation of the occurrence of an anomaly in terms of data that the model previously saw. Even if this explanation is course-grained, it is still useful as the data scientist can use this to trigger further exploratory data analysis to understand the underlying patterns. So perhaps, the models for interpretability being studied in the ML community are more complex than what we need for ML debugging.

Instead, we propose the admittedly imprecise notion of \emph{debugability}.
Let $M$ be a model trained on a dataset of feature and label tuples $(x_i,y_i)$.
Suppose, M sees a new example $x'$ and predicts $y'$ causing a \emph{prediction anomaly} (i.e., incorrect prediction or unsafe output).
Debugability is a measure of how well can we isolate tuples in the training dataset that significantly contributed to the prediction $y'$.

Using this working definition, we can actually design algorithms that take in a black-box model and return a more debugable approximation.
The key insight is that complex models tend to be hierarchical in nature.
One can think of a complex model as a collection of more compartmentalized models (ones that only apply to certain types of examples), and a meta model that selects which one of the sub-models is relevant to the example at hand.
In prior work, we have developed algorithms that can take a set of predictions from a model and infer a likely hierarchical structure~\cite{DBLP:journals/corr/KrishnanGLMPG16, Krishnan17}.
The basic idea is an iterative clustering algorithm that first initializes $k$ models, then assigns tuples to the model that best predicts it, then updates the $k$ models and repeats.
Once the k models are trained, we can then learn a meta-model to switch between them.
When a new example for prediction comes in, the meta-model first picks a sub-model, then the sub-model issues a prediction.
Our initial results have demonstrated that such an approach can improve convergence and stability in control problems.

This paper presents a special case of this framework aimed at improving debugging in large-scale ML settings. 
In particular, we allow the $k$ sub-models to be as expressive and complex as the user desires, but constrain the meta-model to be a decision tree.
This means that if a particular new example $x'$ creates an anomalous output, we can immediately blame a particular sub-model.
Furthermore, the decision tree allows us to quickly determine a predicate to select the subset of training data that contributed to the sub-model.
By varying $k$ and the depth of the decision tree, the user can tradeoff fidelity to the original model and the granularity of such explanations.


Our main justification is that modern ML models are necessarily complex, i.e., they often map complex data structures such as images or documents to decisions. Trying to explain every model in terms of its features, is not a tractable solution with current algorithms. Instead, if we can merely explain the high-level structure of a model (while the sub-models are still opaque), this might be enough of a signal for a data scientist to debug. As an analogy, a high-school physics teacher has to simplify the calculus in his lesson, but leaves enough detail for interested students to learn more on their own.




























\section{Framework and API}
This section describes how we model machine learning prediction and the \sys problem.
Let the training dataset $D$ contain tuples of features $x_i$ and labels $y_i$ that are categorical or real-valued.
\texttt{train(D)} is a training algorithm that returns a model \texttt{model($\cdot$)} that takes as input a test point $x_{new}$ and predicts a label $\hat{y}_{new}$:
\[
\hat{y}_{new} = \textsf{model}(x_{new})
\]

\vspace{0.5em}\noindent \textbf{Example Insurance Fraud Detection: } Consider the following running example of car insurance fraud detection. The training dataset $D$ has the following schema:
\[
\texttt{D(make, amount, at\_fault, descr, fraud?)}
\]
where \texttt{make} is a categorical attribute describing the make of the car, \texttt{amount} is a double-valued amount that the person is claiming, \texttt{at\_fault} is a boolean variable describing whether the claimant is at fault, \texttt{descr} is a string-valued attribute describing the nature of the claim, and \texttt{fraud?} is the yes/no label if the claim is fraudulent.

We can think of a test point as a tuple that is missing a \texttt{label} attribute: 
$$x_{new} = \texttt{(make, amount, at\_fault, descr, \_)}$$
The model fills in the \texttt{label} attribute:
\[
  \texttt{model(}x_{new}\texttt{)} \rightarrow \texttt{(make, amount, at\_fault, descr,} \hat{y}_{new} )
\]
A misprediction is a violation of a constraint that the predicted label is equal to the true label $y^*_{new}$:
\[
  \texttt{model(}x_{new}\texttt{).label = } y^*_{new}
\]

\subsection{Challenges in Debugging}
Suppose $\textsf{model}(\cdot)$ issues an incorrect prediction and erroneously flags a fraudulent claim as not fraudulent. The data scientist is now tasked with debugging the model to understand why this error happened. 
She considers three possibilities: 
\begin{itemize}
\item \emph{(P1) Model Error.} The model is not sufficiently accurate enough to predict all claims correctly and needs to be tuned to err on the side of false positives.
\item \emph{(P2) Approximation Error.} The model was not trained with sufficient examples that look like $x_{new}$, and therefore, is not accurate in that region of the feature space.
\item \emph{(P3) Data Error.} The record $x_{new}$ is not consistent with respect to the training data, i.e., it represents the same information differently--leading to an unpredictable featurization.
\end{itemize}
The challenge for the data scientist is to determine which of these categories of errors best describes why the claim was mispredicted.
Intuitively, her process is to look at how ``similar'' tuples in the training dataset were predicted and compare to the given record. This will allow her to evaluate whether the problem is inherent to the model class or due to a fault of the training dataset or data processing pipeline. 

Existing approaches: 
\begin{itemize}
\item \emph{(S1) Nearest Neighbors. } The data scientist can search for the k nearest neighbors in the training dataset and use those as a guide.

\item \emph{(S2) Use A Simpler Model.} The data scientist can use any number of techniques to use the interpretable model from the start, then apply her domain expertise to understand why an error occurred.
\end{itemize}

The problem with (S1) and (S2) is that the prediction problem necessarily needs complex models. In the running example, there is a textual field \textsf{descr}, which might be a very valuable feature for predicting fraud. Processing such data typically requires translating the data into a higher-dimensional feature space by using NLP methods such as word-embeddings, stop word removal, bi-gram featurization, etc.  By design, this feature space may not be interpretable by anyone other than a language expert, and using a simpler model may not achieve the desired accuracy. Furthermore, highly expressive deep models such as deep neural networks, which in principle can learn any deterministic function, are susceptible to adversarial examples (i.e., imperceptible perturbations to the features that cause a change in prediction)\cite{szegedy2013intriguing}. This means that relying on neighboring points can be unreliable or even misleading because they may inadvertantly be adversarial.  The approach that we propose identifies records that the {\it classifier} considers similar, rather than naive similarity in the feature space.

\subsection{Debugging with \sys}
Our goal is to isolate points in the training data that significantly contributed to a anomalous prediction $\hat{y}_{new}$.
Given a model $\textsf{model}(\cdot)$, we want to train a surrogate \sys model $\textsf{smodel}(\cdot)$ that approximates the original model, but can help the developer identify such points.
The structure of a \sys model is a meta-model that partitions the training data, and a local sub-model for each partition that is {\it Partition Aware}.
\sys models can be viewed as a generalization of piecewise linear models to higher dimensional spaces, and for arbitrary non-linear models.     
In this paper, we focus on decision trees for the meta-model, and the same model class as \texttt{model} for the sub-models, however, it may be possible to exhibit different accuracy and explanatory behaviors by varying the class of models we use for the meta-model and sub-models.

As an illustrative example, consider when the meta model is much simpler.
For example, imagine if we hard-coded the following logic:
\begin{lstlisting}
def smodel(x):
    if amount > 10000:
    #one model for larger claims
        return submodel_1(x)
    elif a_fault:
    #one model for small at fault claims
        return submodel_2(x)
    else:
    #a default model
        return submodel_3(x)
\end{lstlisting}
Even though meta model is simple (in fact it is a decision tree), the full model \texttt{smodel} can still model complex patterns due to the complex sub-models.  Thus, we do not sacrafice model accuracy in the same way as simpler model approximations.
But, if we observe an anomaly, the meta-model can precisely blame one of the submodels; thereby, providing a coarse predicate to select tuples that are assigned to that model.
There is an inherent tradeoff, a meta-model that contains a single partition \texttt{true} can create a sub-model that is nearly identical to the original complex model, however the partition is not useful to the developer.  In contrast, increasing the complexity of the meta-model makes each partition smaller, but the resulting model potentially diverges from the orginal model.  At the limit, the meta-model may simply build a decision tree over the training data, and each partition contains nearly identical points.

\subsection{Learning Meta Models} 
We propose an algorithm to automatically learn a meta-model and submodels from data to approximate the user's desired model.
This algorithm can run offline during the training phase and is generally no-more than a constant factor more expensive than standard model training
This algorithm is a special case of the algorithm, we proposed in prior work~\cite{DBLP:journals/corr/KrishnanGLMPG16, krishnan17}. At a high-level, the algorithm takes the dataset $D$ and the model \textsf{model} as input, and returns $\textsf{smodel}$, which consists of a decision-tree meta model that selects from a collection of $k$ submodels. Given a new record, the user can evaluate both:
\[
\hat{y}_{new} = \textsf{model}(x_{new})
\]
\[
\hat{y}_{new} = \textsf{smodel}(x_{new}) \approx \textsf{model}(x_{new})
\]
and use the structure of \textsf{smodel} to debug with knowledge that it approximates $\textsf{model}$.

\subsection{API and System}
We implemented this as a system in Python initially focusing on TensorFlow models.
For the user, the inputs are:
\begin{itemize}
\item \emph{Featurized Dataset. } The user provides a dataset of feature and label tuples.

\item \emph{Explainable Features. } The user lists a subset of features that are understandable.

\item \emph{Tensorflow Model Description. } The user provides a symbolic description of the model in Tensorflow.

\item \emph{Number of Sub-models. } The user provides the number of submodels to include in the surrogate model (denoted as $k$).
\end{itemize}

The output of the system is:
\begin{itemize}
\item \emph{Original Model. } The original model trained to completion

\item \emph{K Sub-Models. } The system returns K submodels trained on different partitions of the feature-space.

\item \emph{Decision Tree Meta Model. } The system returns a meta model that switches between the K submodels based on the input record.
\end{itemize}

We implemented the algorithm into a web interface that allows users to debug TensorFlow models Figure \ref{fig:interface}. The interface shows users mispredictions and allows them to search records that the classifier treats as similar.

\begin{figure*}[t]
    \centering
    \includegraphics[width=0.6\textwidth]{figures/interface.png}
    \caption{A prototype interface implementing the algorithm. In (A), the interface lists a set tuples that were mis-predicted. Users can dig deeper by selecting one such tuple. (B) is the following panel which describes a predicate and records that the classifier considers as similar.}
    \label{fig:interface}
\end{figure*}
\section{Algorithm Description}
In prior work, we designed an algorithm in a completely different context---to decompose complex control policies to reduce planning horizons~\cite{DBLP:journals/corr/KrishnanGLMPG16, krishnan17}. Surprisingly, when we started discussing our ideas with ML developers, we realized a very similar approach could apply for problems in debugging. 

For intuition on how the algorithm works, consider a standard KMeans clustering of dataset. First, $k$ random cluster centers are placed over the feature-space. Then, data are assigned to the nearest cluster. Finally, the clusters are updated based on the assignment, and the algorithm repeats by re-assigning data based on the most recent update.

Similarly, we can do the same thing for model training. $k$ random sub-models are initialized.
During the ``assignment'' step, a data point is assigned to the sub-model that best predicts it.
Then, during the update step, the models are updated based on the new assignments with gradient descent.
This is a variant of the popular Expectation-Maximization algorithm, that instead of a Maximization step computes a gradient instead. 

This algorithm to run efficiently at scale and directly integrate with TensorFlow's Python API.
So, at training time (offline) the user can not only train the original model but also the surrogate.
We also integrated the algorithm into a local web interface that visualized the results.


\subsection{Technical Details}

\vspace{0.5em} \noindent \textbf{Fitting Step: } To construct $\textsf{smodel}(\cdot)$ from $\textsf{model}(\cdot)$, we first start off by fitting the parameters to a simplified probabilistic model. 
The first step is to run $\textsf{model}(\cdot)$ over the entire training dataset and get feature-prediction $(x, \hat{y})$ tuples.
We can define $f(\hat{y} \mid x)$ which is the probability of the label $\hat{y}$ given the feature $x$ to represent how $\textsf{model}(\cdot)$ generates predictions. 
Arbitrary probability distributions are hard to reason about so we consider parametrized distributions $f(\hat{y} \mid x, \theta)$. 
We want to find $k$ such distributions that best explain all the observations:
\[
\{f(\hat{y} \mid x, \theta_1),...,f(\hat{y} \mid x, \theta_k)\}
\]
Our results in prior work~\cite{krishnan17} show how this can be optimized with a two-step algorithm:
\begin{itemize}
    \item Initialize $\theta_1,...,\theta_k$ randomly.
    \item Repeat until convergence
    \item For each data point $i \in \{1,...,N\}$: 
    \item \begin{itemize} 
          \item For each component distribution $j \in \{1,...,k\}$:
          \item $w(i,j) = f(\hat{y}_i \mid x_i, \theta_j)$
    \end{itemize}
    \item Gradient Ascent for each $\theta$:
    \[ \theta_j \leftarrow \theta_j + \lambda \sum_i^N w(i,j) \nabla \log f(\hat{y}_i \mid x_i, \theta_j)  \]
\end{itemize}

The intuition behind this algorithm is that after random initialization the initial models will have higher accuracy on different data points (by chance). $w(i,j)$ is a soft-assignment--assigning data points to the model that best explains its prediction. Then, $w(i,j)$ becomes a weight to update the models with Gradient Ascent (or descent over the negative log likelihood). This process repeats until convergence. This is very similar to the K-Means or EM algorithm, but instead of updating the cluster centers with a formula, we take a gradient step.

\vspace{0.5em} \noindent \textbf{Distillation Step: } The result of the first step is a set of model parameters $k$ $\theta_1,...,\theta_k$, and a weighting function $w(i,j)$. The next step is to distill $w(i,j)$ into a set of explainable rules that select the one of the $k$ models. We first generate a set of hard assignments for each data point:
\[
h(i) = \arg\max_{j} w(i,j)
\]
Each $h$ is an indicator $1,...,k$ of the most likely assignment of each data point.
We can now train a more explainable model to select $h$ as a function of the data.
This intuition is that this is a simpler model that just selects one of the component models.

The user provides us with a list of features that are considered ``explainable'', and let $x_e$ denote the projection of an example onto this subset of features.
Then, we can train a decision tree over the tuples $(x_e, h)$, which have the desirable property of resembling programmatic statements.
This decision tree is a multi-class classifier that predicts the assignment to a submodel as a function of the intrepretable features.
We call this model the meta-model, as it selects between the component models.
Finally, putting everything together, we return something that looks similar to the hard-coded example in the previous section.
The surrogate model $\textsf{smodel}(\cdot)$ encapsulates submodels that apply in different parts of the feature-space.
Suppose, we observe an anomaly, we can now blame a specific model and efficiently get a predicate that selects all of the data the contributed to the model.
\input{interface.tex}

\begin{figure*}[ht]
    \centering
    \includegraphics[width=\textwidth]{figures/result1.png}
    \caption{(A) Illustrates how well a debuggable surrogate model can approximate a complex model, (B) illustrates how well this surrogate model can isolate mispredictions in terms of precision and recall, and (C) why alternative approaches do not work.}
    \label{fig:teaser}
\end{figure*}

\section{Highlighted Results}
We present an illustrative example in this paper to describe its applications to supervised learning.
We have a dataset of movie descriptions IMDB~\footnote{ \url{ftp://ftp.fu-berlin.de/pub/misc/movies/database/}} and Yahoo~\footnote{ \url{http://webscope.sandbox.yahoo.com/catalog.php?datatype=r}}.
Each movie has a title, a short 1-2 paragraph plot description, year, rating, language, and a list of categories, and the goal is to train a model to predict whether a movie is a ``Horror'' or ``Comedy'' from the description and title.  
The total dataset has 506,244 records.

First, using TensorFlow, we trained a LSTM-based model to predict these categories. The first layer of this model computes what is called a word-embedding, where the LSTM learns a feature-space in which similar words (co-occuring) are closer together. 
The next two layers consist of dense layers that map the words from the feature-space to classification outputs.
The result is a model that achieves 93\% accuracy, which is far more accurate than simpler alternatives on a Bag-of-Words featurization (random forests 90\%, Linear SVM 81\%, Kernel SVM 85\%).
The problem is that it is hard to diagnose what this model is exactly doing, unlike the simpler alternatives. 

We applied our approach to learn a surrogate model that approximates the original one (Figure \ref{fig:teaser}). We varied the number of submodels $k$ to illustrate the tradeoff. Figure \ref{fig:teaser}a shows the agreement between the surrogate and the original model as a function of $k$. As the model is increasingly compartmentalized with a higher $k$ the agreement goes down, however, not too drastically. The next plot shows how well the surrogate model can isolate related records (Figure \ref{fig:teaser}b). We randomly sample 10 mispredictions of the model, and identify the submodel that mispredicted those examples. We then query all of the data that contributed to that submodel and measure how effective that submodel is at extracting other mispredictions (in terms of precision and recall).
As $k$ increases, we see an increased ability to isolate errors, but the surrogate increasingly deviates from the original model.
Finally, we compare the algorithm to baselines to show that it is far more accurate than random guessing or a simple k-Nearest Neighbor search in TFIDF space (Figure \ref{fig:teaser}c).



\section{Recent Work and Next Steps}
The study of model intepretability and explainability is recently a hot topic in ML research~\cite{taylor2016alignment, lei2016rationalizing, ribeiro2016should}, especially in the context of Neural Networks. 
An example of one such approach is to train a sparse linear model in the local neighborhood of a point\cite{lei2016rationalizing}.
Another more recent approach in computer vision is to use attention models, enforce that the model focuses on certain features, for explainability~\cite{kim2017interpretable}.
Another relevant line of work is Neural Network Rule Extraction~\cite{hailesilassie2016rule}.
This problem is very challenging since highly expressive deep models such as deep neural networks, which in principle can learn any deterministic function, are susceptible to adversarial examples (i.e., imperceptible perturbations to the features that cause a change in prediction)\cite{szegedy2013intriguing}.
Explaining a prediction exactly in terms of features is highly useful for an end-user, but developers also need to be able to trace modeling errors to training data~\cite{DBLP:journals/pvldb/KrishnanWWFG16}.
This is why our approach focuses on identifying the most informative neighborhood (partition) of training data.
We believe that techniques that isolate features and data are complementary and hope to explore combinations of the two in the future.

Based on our initial study and survey of recent related work, we have highlighted a number of important challenges for the future systems.

\vspace{0.5em}\noindent\textbf{Connecting Explainability to Data Provenance: } We believe that there is further work to be done to explain predictions in terms of relevant source data. Systems like \sys can be connected to lineage systems to trace even further upstream than just the training data. This leads a number of computational challenges in storing, processing, and summarizing the selected tuples.

\vspace{0.5em}\noindent\textbf{Reducing Hyper-Parameters and Failure Modes: } Ironically, existing work in explainable models, including \sys, all have subtle failure modes due to their assumptions and hyper-parameters. This could lead to faulty or misleading explanations and erode user trust. We hope to explore techniques that require less tuning and less detailed understanding of the mathematical structure of the problem in the future.

\vspace{0.5em}\noindent\textbf{Scalability of Human Effort: } Finally, an important concern is how human analysts can explore and iterate through large training datasets. While systems like \sys can reduce the burden, there are still many more hurdles before truly useful machine learning debuggers. We believe that coupling explanations with anomaly detection may be a viable next step, as in the MacroBase project~\cite{bailis2017macrobase}.















%\input{problem_statement.tex}
%\input{architecture.tex}
%\input{naive.tex}
%\input{detector.tex}
%\input{sampling.tex}
%\input{estimator.tex}
%\input{optimal.tex}
% \input{detect.tex}

% \input{impestimate.tex}

%\input{optimizer.tex}
%\input{experiments.tex}
%\section{Recent Work and Next Steps}
The study of model intepretability and explainability is recently a hot topic in ML research~\cite{taylor2016alignment, lei2016rationalizing, ribeiro2016should}, especially in the context of Neural Networks. 
An example of one such approach is to train a sparse linear model in the local neighborhood of a point\cite{lei2016rationalizing}.
Another more recent approach in computer vision is to use attention models, enforce that the model focuses on certain features, for explainability~\cite{kim2017interpretable}.
Another relevant line of work is Neural Network Rule Extraction~\cite{hailesilassie2016rule}.
This problem is very challenging since highly expressive deep models such as deep neural networks, which in principle can learn any deterministic function, are susceptible to adversarial examples (i.e., imperceptible perturbations to the features that cause a change in prediction)\cite{szegedy2013intriguing}.
Explaining a prediction exactly in terms of features is highly useful for an end-user, but developers also need to be able to trace modeling errors to training data~\cite{DBLP:journals/pvldb/KrishnanWWFG16}.
This is why our approach focuses on identifying the most informative neighborhood (partition) of training data.
We believe that techniques that isolate features and data are complementary and hope to explore combinations of the two in the future.

Based on our initial study and survey of recent related work, we have highlighted a number of important challenges for the future systems.

\vspace{0.5em}\noindent\textbf{Connecting Explainability to Data Provenance: } We believe that there is further work to be done to explain predictions in terms of relevant source data. Systems like \sys can be connected to lineage systems to trace even further upstream than just the training data. This leads a number of computational challenges in storing, processing, and summarizing the selected tuples.

\vspace{0.5em}\noindent\textbf{Reducing Hyper-Parameters and Failure Modes: } Ironically, existing work in explainable models, including \sys, all have subtle failure modes due to their assumptions and hyper-parameters. This could lead to faulty or misleading explanations and erode user trust. We hope to explore techniques that require less tuning and less detailed understanding of the mathematical structure of the problem in the future.

\vspace{0.5em}\noindent\textbf{Scalability of Human Effort: } Finally, an important concern is how human analysts can explore and iterate through large training datasets. While systems like \sys can reduce the burden, there are still many more hurdles before truly useful machine learning debuggers. We believe that coupling explanations with anomaly detection may be a viable next step, as in the MacroBase project~\cite{bailis2017macrobase}.














%\input{discussion.tex}
%\input{conclusion.tex}
%\input{outlier.tex}
%\input{analysis.tex}
%\input{experiments.tex}
%\input{conclusion.tex}


%\bibliographystyle{abbrv}
%\scriptsize

\vspace{0.5em}
\noindent \textbf{Acknowledgements: }
This research was supported in part by a seed grant from the UC Berkeley Center for Information Technology in the Interest of Society (CITRIS), the UC Berkeley RISELab, and by the U.S. National Science Foundation under Award IIS-1227536: Multilateral Manipulation by Human-Robot Collaborative Systems. This work has been supported in part by funding from Google and and Cisco.


%\fontsize{8.8pt}{9.9pt} \selectfont
\bibliographystyle{ACM-Reference-Format}
\bibliography{ref,explainability} 
\normalsize \selectfont
%\appendix
%\input{appendix.tex}

\end{document}
